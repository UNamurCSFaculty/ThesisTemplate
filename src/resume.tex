
Le génie des Lignes de Produits Logiciels (LPL) est une sous-discipline du génie logiciel basée sur l'idée que les produits d'une même famille peuvent être construits de manière systématique en réutilisant des briques de base. Certaines sont communes à tous les produits de la famille et certaines sont spécifiques à un sous-ensemble de ces produits. Depuis la création de la discipline, le génie des LPLs s'intéresse à la question du test d'une LPL. Le grand nombre de produits possibles pouvant être dérivés d'une LPL amène un nombre encore plus important de cas de test. Les recherches précédentes se sont focalisées sur la réutilisation (d'une partie de) ces cas de test d'un produit à l'autre. Afin de ne pas devoir tester tous les produits possibles, des techniques d'échantillonnage produisent un sous-ensemble représentatif de produits à tester en priorité. Ces techniques raisonnent au niveau de la famille de produits en se basant sur le modèle de variabilité de la LPL. Cependant, elles ne prennent pas en compte d'autres aspects. Par exemple, le comportement des produits.

Dans cette thèse, nous présentons une infrastructure pour effectuer du test de LPL dirigé par les modèles, en raisonnant au niveau de la famille de produits. Nous utilisons des \textit{\acrfullpl{FTS}}, un formalisme compact pour représenter le comportement d'une LPL dans son ensemble, afin d'effectuer les différentes activités de test. La sélection et la priorisation de cas de test se font sur base du comportement de la LPL et sont dirigées par trois types de critères de couverture : les critères basés sur la structure du FTS, les critères basés sur la dissimilarité des cas de test et les critères statistiques basés sur l'utilisation effective des produits.
Nous effectuons également une analyse des cas en test en utilisant la mutation et notre \textit{\acrfull{FMM}}. Cette analyse comprend une détection des mutants équivalents. 
Le résultat d'un processus de sélection est un ensemble de cas de test, définis pour la famille de produits et pouvant servir à définir les produits à tester en priorité.

L'approche a été implémentée dans un \textit{\acrfull{VIBeS} framework} open-source et évaluée sur différents cas d'étude de différentes tailles, représentant des systèmes embarqués et des applications Web. Les résultats démontrent l'applicabilité de l'approche pour sélectionner des cas de test et effectuer une analyse basée sur la mutation. Ils confirment la pertinence de combiner modèles de comportement et de variabilité pour améliorer le test de LPL dirigé par les modèles et la mutation.

\textbf{Mots clés :} test dirigé par les modèles, lignes de produits logiciels, test logiciel
